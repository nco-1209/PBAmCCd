\nonstopmode{}
\documentclass[a4paper]{book}
\usepackage[times,inconsolata,hyper]{Rd}
\usepackage{makeidx}
\usepackage[utf8]{inputenc} % @SET ENCODING@
% \usepackage{graphicx} % @USE GRAPHICX@
\makeindex{}
\begin{document}
\chapter*{}
\begin{center}
{\textbf{\huge Package `PBAmCCd'}}
\par\bigskip{\large \today}
\end{center}
\ifthenelse{\boolean{Rd@use@hyper}}{\hypersetup{pdftitle = {PBAmCCd: Proportionality Based Association Measures for Count Compositional Data}}}{}
\begin{description}
\raggedright{}
\item[Type]\AsIs{Package}
\item[Title]\AsIs{Proportionality Based Association Measures for Count
Compositional Data}
\item[Author]\AsIs{Kevin McGregor, Nneka Okaeme}
\item[Maintainer]\AsIs{Kevin McGregor }\email{kevinmcg@yorku.ca}\AsIs{}
\item[Description]\AsIs{Package corresponding to the paper ``Proportionality Based Association 
Measures for Count Compositional Data''.}
\item[License]\AsIs{Artistic-2.0}
\item[Version]\AsIs{0.0.1.0}
\item[RoxygenNote]\AsIs{7.2.1}
\end{description}
\Rdcontents{\R{} topics documented:}
\inputencoding{utf8}
\HeaderA{dirichletVariation}{Dirichlet Variation}{dirichletVariation}
%
\begin{Description}\relax
Estimates the variation matrix for empirical compositional data which is based
on the Dirichlet distribution.
\end{Description}
%
\begin{Usage}
\begin{verbatim}
dirichletVariation(counts)
\end{verbatim}
\end{Usage}
%
\begin{Arguments}
\begin{ldescription}
\item[\code{counts}] Dataset of compositional data
\end{ldescription}
\end{Arguments}
%
\begin{Value}
\code{NULL}
\end{Value}
\inputencoding{utf8}
\HeaderA{ebic}{Extended Bayesian Information Criterion}{ebic}
%
\begin{Description}\relax
Calculates the Extende
d Bayesian Information Criterion (EBIC) of a model.
Used for model selection.
\end{Description}
%
\begin{Usage}
\begin{verbatim}
ebic(l, n, d, df, gamma)
\end{verbatim}
\end{Usage}
%
\begin{Arguments}
\begin{ldescription}
\item[\code{l}] Log-likelihood estimates of the model

\item[\code{n}] Number of rows of the data set for which the log-likelihood has been 
calculated

\item[\code{d}] ?

\item[\code{df}] Degrees of freedom

\item[\code{gamma}] ?
\end{ldescription}
\end{Arguments}
%
\begin{Value}
The value of the EBIC.
\end{Value}
\inputencoding{utf8}
\HeaderA{ebicPlot}{Extended Bayesian Information Criterion Plot}{ebicPlot}
%
\begin{Description}\relax
Plots the extended Bayesian information criterion (EBIC) of the model fit for
various penalization parameters \code{lambda}.
\end{Description}
%
\begin{Usage}
\begin{verbatim}
ebicPlot(fit, xlog = FALSE)
\end{verbatim}
\end{Usage}
%
\begin{Arguments}
\begin{ldescription}
\item[\code{fit}] The model fit

\item[\code{xlog}] TRUE or FALSE. Renders plot with the x-axis in the log-scale if TRUE
\end{ldescription}
\end{Arguments}
%
\begin{Value}
Plot of the EBIC (y-axis) at each lambda (x-axis).
\end{Value}
\inputencoding{utf8}
\HeaderA{g}{Log of the Inverse Additive Log-ratio}{g}
%
\begin{Description}\relax
Function which takes the logarithm of a vector, after the inverse additive 
log-ratio transformation has been applied to it.
\end{Description}
%
\begin{Usage}
\begin{verbatim}
g(x)
\end{verbatim}
\end{Usage}
%
\begin{Arguments}
\begin{ldescription}
\item[\code{x}] Compositional data vector which has already been transformed by the 
additive log-ratio
\end{ldescription}
\end{Arguments}
%
\begin{Value}
A vector which is the log of the inverse of the data which has been
transformed by the additive logratio transformation.
\end{Value}
%
\begin{Examples}
\begin{ExampleCode}
vec <- sample(1:250, 9)
ref <- vec[length(vec)]
alr.vec <- log(vec[-length(vec)]/ref)

g(alr.vec)

\end{ExampleCode}
\end{Examples}
\inputencoding{utf8}
\HeaderA{grad}{Gradient of Normal Random Variables}{grad}
%
\begin{Description}\relax
Calculates the gradient of the normal random variables, on the logit scale.
\end{Description}
%
\begin{Usage}
\begin{verbatim}
grad(v, y, ni, mu, Sigma.inv)
\end{verbatim}
\end{Usage}
%
\begin{Arguments}
\begin{ldescription}
\item[\code{v}] Additive-log ratio (alr) transform of y

\item[\code{y}] Compositional data set

\item[\code{ni}] Row sums of y?

\item[\code{mu}] Mu vector of y

\item[\code{Sigma.inv}] Sigma inverse matrix of y
\end{ldescription}
\end{Arguments}
%
\begin{Value}
The gradient vector of Normal random variables
\end{Value}
\inputencoding{utf8}
\HeaderA{hess}{Hessian}{hess}
%
\begin{Description}\relax
Calculates the hessian matrix.
\end{Description}
%
\begin{Usage}
\begin{verbatim}
hess(v, ni, Sigma.inv)
\end{verbatim}
\end{Usage}
%
\begin{Arguments}
\begin{ldescription}
\item[\code{v}] Compositional dataset which has been transformed by the additive log ratio

\item[\code{ni}] Row sums of the raw data ?

\item[\code{Sigma.inv}] Inverse of the Sigma matrix
\end{ldescription}
\end{Arguments}
%
\begin{Value}
The hessian matrix.
\end{Value}
\inputencoding{utf8}
\HeaderA{logitNormalVariation}{Logit Normal Variation}{logitNormalVariation}
%
\begin{Description}\relax
Estimates the empirical variation matrix of count data that follows a multinomial
logit-Normal distribution.
\end{Description}
%
\begin{Usage}
\begin{verbatim}
logitNormalVariation(
  mu,
  Sigma,
  type = c("standard", "phi", "phis", "rho"),
  order = c("first", "second")
)
\end{verbatim}
\end{Usage}
%
\begin{Arguments}
\begin{ldescription}
\item[\code{mu}] The mle estimate of the mu matrix

\item[\code{Sigma}] The mle estimate of the sigma matrix

\item[\code{type}] Type of variation metric to be calculated: \code{standard}, \code{phi},
\code{phis} (a symmetrical version of \code{phi}), or \code{rho}

\item[\code{order}] The order of the Taylor-series approximation to be used in the 
estimation
\end{ldescription}
\end{Arguments}
%
\begin{Value}
An estimation of the variation matrix, \code{V}.
\end{Value}
\inputencoding{utf8}
\HeaderA{logitNormalVariationOLD}{Logit Normal Variation (Old Version)}{logitNormalVariationOLD}
%
\begin{Description}\relax
Estimates the empirical variation matrix of count data that follows a multinomial 
logit-Normal distribution. Differs from the function \code{logitNormalVariation()}
in that this function does not use a Taylor-series approximation.
\end{Description}
%
\begin{Usage}
\begin{verbatim}
logitNormalVariationOLD(
  mu,
  Sigma,
  lmu,
  lsigma,
  type = c("standard", "phi", "rho")
)
\end{verbatim}
\end{Usage}
%
\begin{Arguments}
\begin{ldescription}
\item[\code{mu}] ?

\item[\code{Sigma}] ?

\item[\code{lmu}] ?

\item[\code{lsigma}] ?

\item[\code{type}] Type of variation metric to be calculated: \code{standard}, \code{phi},
or \code{rho}
\end{ldescription}
\end{Arguments}
%
\begin{Value}
An estimation of the variation matrix, \code{V}.
\end{Value}
\inputencoding{utf8}
\HeaderA{logLik}{Log-Likelihood}{logLik}
%
\begin{Description}\relax
Calculates the log-likelihood.
\end{Description}
%
\begin{Usage}
\begin{verbatim}
logLik(v, y, ni, S, invSigma)
\end{verbatim}
\end{Usage}
%
\begin{Arguments}
\begin{ldescription}
\item[\code{v}] The additive log ratio transform of y

\item[\code{y}] Compositional data set

\item[\code{ni}] The row sums of y?

\item[\code{S}] ?

\item[\code{invSigma}] The inverse of the Sigma matrix
\end{ldescription}
\end{Arguments}
%
\begin{Value}
The estimated log-likelihood under the Multinomial Logit-Normal distribution.
\end{Value}
\inputencoding{utf8}
\HeaderA{logLikG}{Gaussian Log-Likelihood}{logLikG}
%
\begin{Description}\relax
Calculates the Gaussian log-likelihood.
\end{Description}
%
\begin{Usage}
\begin{verbatim}
logLikG(v, S, invSigma)
\end{verbatim}
\end{Usage}
%
\begin{Arguments}
\begin{ldescription}
\item[\code{v}] Compositional data set which has been transformed by the additive 
log ratio

\item[\code{S}] ?

\item[\code{invSigma}] Inverse of the Sigma matrix
\end{ldescription}
\end{Arguments}
%
\begin{Value}
The estimated Gaussian log-likelihood under the Multinomial Logit-Normal distribution.
\end{Value}
\inputencoding{utf8}
\HeaderA{logVarMC}{Monte Carlo Logx Variance-Covariance}{logVarMC}
%
\begin{Description}\relax
Function which estimates the variance0covariance of the log of the data, using Monte
Carlo integration.
\end{Description}
%
\begin{Usage}
\begin{verbatim}
logVarMC(mu, Sigma, K = 1e+05)
\end{verbatim}
\end{Usage}
%
\begin{Arguments}
\begin{ldescription}
\item[\code{mu}] The mean vector of the underlying distribution

\item[\code{Sigma}] The variance matrix of the underlying distribution

\item[\code{K}] Number of samples
\end{ldescription}
\end{Arguments}
%
\begin{Value}
The estimated variance-covariance matrix, \code{logx}
\end{Value}
\inputencoding{utf8}
\HeaderA{logVarTaylor}{Logx Variance-Covariance}{logVarTaylor}
%
\begin{Description}\relax
Function which estimates the variance-covariance of the log of the data, using a 
Taylor-series approximation.
\end{Description}
%
\begin{Usage}
\begin{verbatim}
logVarTaylor(mu, Sigma, transf = c("alr", "clr"), order = c("first", "second"))
\end{verbatim}
\end{Usage}
%
\begin{Arguments}
\begin{ldescription}
\item[\code{mu}] The mean vector of the underlying distribution

\item[\code{Sigma}] The sigma matrix of the underlying distribution

\item[\code{transf}] The desired transformation. If \code{transf="alr"} the inverse 
additive log-ratio transformation is applied. If \code{transf="clr"} the
inverse centered log-ratio transformation is applied.

\item[\code{order}] The desired order of the Taylor Series approximation
\end{ldescription}
\end{Arguments}
%
\begin{Value}
The estimated variance-covariance matrix, \code{logx}.
\end{Value}
\inputencoding{utf8}
\HeaderA{logVarTaylorFull}{Full Logx Variance-Covariance}{logVarTaylorFull}
%
\begin{Description}\relax
Function which estimates the variance-covariance of the log of the data, using a 
Taylor-series approximation. This function differs from \code{Logx Variance} in
that the resultant matrix includes a reference category.
\end{Description}
%
\begin{Usage}
\begin{verbatim}
logVarTaylorFull(
  mu,
  Sigma,
  transf = c("alr", "clr"),
  order = c("first", "second")
)
\end{verbatim}
\end{Usage}
%
\begin{Arguments}
\begin{ldescription}
\item[\code{mu}] The mean vector of the underlying distribution

\item[\code{Sigma}] The sigma matrix of the underlying distribution

\item[\code{transf}] The desired transformation. If \code{transf="alr"} the inverse 
additive logratio transformation is applied. If \code{transf="clr"} the
inverse centered logratio transformation is applied.

\item[\code{order}] The desired order of the Taylor Series approximation
\end{ldescription}
\end{Arguments}
%
\begin{Value}
The estimated variance-covariance matrix, \code{logx}.
\end{Value}
\inputencoding{utf8}
\HeaderA{logVarUnscented}{Unscented Logx Variance-Covariance}{logVarUnscented}
%
\begin{Description}\relax
Function which estimates the variance-covariance of the log of the data, using an 
unscented transformation.
\end{Description}
%
\begin{Usage}
\begin{verbatim}
logVarUnscented(
  mu,
  Sigma,
  transf = c("alr", "clr"),
  alpha = 0.001,
  beta = 2,
  kappa = 0
)
\end{verbatim}
\end{Usage}
%
\begin{Arguments}
\begin{ldescription}
\item[\code{mu}] The mean vector of the underlying distribution

\item[\code{Sigma}] The sigma matrix of the underlying distribution

\item[\code{transf}] The desired transformation. If \code{transf="alr"} the inverse 
additive log-ratio transformation is applied. If \code{transf="clr"} the
inverse centered log-ratio transformation is applied.

\item[\code{alpha}] Parameter which controls the spread of the Sigma points

\item[\code{beta}] Parameter which compensates for the distribution

\item[\code{kappa}] Scaling Parameter
\end{ldescription}
\end{Arguments}
%
\begin{Value}
The estimated variance-covariance matrix, logx.
\end{Value}
%
\begin{References}\relax
Extended and Unscented Kalman Filter Algorithms for Online State Estimation. 
(2022). MathWorks. https://www.bibliography.com/apa/apa-reference-page-examples-and-format-guide/

Hendeby, G., \& Gustafsson, F. On Nonlinear Transformations Of Gaussian Distributions. 
https://users.isy.liu.se/en/rt/fredrik/reports/07SSPut.pdf
\end{References}
\inputencoding{utf8}
\HeaderA{logVarUnscentedW}{Unscented Logx Variance-Covariance   ???What is W for?}{logVarUnscentedW}
%
\begin{Description}\relax
Function which estimates the variance-covariance of the log of the data, using an 
unscented transformation.
\end{Description}
%
\begin{Usage}
\begin{verbatim}
logVarUnscentedW(mu, Sigma, transf = c("alr", "clr"))
\end{verbatim}
\end{Usage}
%
\begin{Arguments}
\begin{ldescription}
\item[\code{mu}] The mean vector of the underlying distribution

\item[\code{Sigma}] The sigma matrix of the underlying distribution

\item[\code{transf}] The desired transformation. If \code{transf="alr"} the inverse 
additive logratio transformation is applied. If \code{transf="clr"} the
inverse centered logratio transformation is applied.
\end{ldescription}
\end{Arguments}
%
\begin{Value}
The estimated variance-covariance matrix, logx.
\end{Value}
\inputencoding{utf8}
\HeaderA{MCSample}{Monte Carlo Sample}{MCSample}
%
\begin{Description}\relax
Generates a Monte Carlo sample based on the multinomial logit-normal model.
\end{Description}
%
\begin{Usage}
\begin{verbatim}
MCSample(mu, Sigma, K = 1)
\end{verbatim}
\end{Usage}
%
\begin{Arguments}
\begin{ldescription}
\item[\code{mu}] Mean matrix of the underlying distribution

\item[\code{Sigma}] Variance matrix of the underlying distribution

\item[\code{K}] Number of samples to generate
\end{ldescription}
\end{Arguments}
%
\begin{Value}
K samples from the multinomial logit-normal model. The number of features
in the sample is of length(\code{mu})+1.
\end{Value}
\inputencoding{utf8}
\HeaderA{MCVariation}{Monte Carlo Variation}{MCVariation}
%
\begin{Description}\relax
Calculates the variation matrix using Monte Carlo integration.
\end{Description}
%
\begin{Usage}
\begin{verbatim}
MCVariation(
  mu = NULL,
  Sigma = NULL,
  x = NULL,
  K = 1e+06,
  type = c("standard", "phi", "phis", "rho", "logx")
)
\end{verbatim}
\end{Usage}
%
\begin{Arguments}
\begin{ldescription}
\item[\code{mu}] The mean matrix of the underlying distribution

\item[\code{Sigma}] The variance matrix of the underlying distribution

\item[\code{x}] A sample from the multinomial logit-Normal model. If \code{mu=NULL} 
and \code{Sigma=NULL}, then x must be provided

\item[\code{K}] Number of Monte Carlo samples to generate

\item[\code{type}] Type of variation metric to be calculated: \code{standard}, \code{phi},
\code{phis} (a symmetrical version of \code{phi}), \code{rho}, or \code{logx}
\end{ldescription}
\end{Arguments}
%
\begin{Value}
The variance matrix, \code{v}.
\end{Value}
\inputencoding{utf8}
\HeaderA{mleLR}{Maximum Likelihood Estimate}{mleLR}
%
\begin{Description}\relax
Function which returns maximum likelihood estimates of parameters given a compositional
dataset. The MLE procedure is based on the Multinomial Logit-Normal distribution, 
using the EM algorithm from Hoff (2003).
\end{Description}
%
\begin{Usage}
\begin{verbatim}
mleLR(
  y,
  max.iter = 10000,
  max.iter.nr = 100,
  tol = 1e-06,
  tol.nr = 1e-06,
  lambda.gl = 0,
  gamma = 0.1
)
\end{verbatim}
\end{Usage}
%
\begin{Arguments}
\begin{ldescription}
\item[\code{y}] Compositional dataset

\item[\code{max.iter}] Maximum number of iterations

\item[\code{max.iter.nr}] Maximum number of Newton-Raphson iterations

\item[\code{tol}] Stopping rule

\item[\code{tol.nr}] Stopping rule for the Newton-Raphson algorithm

\item[\code{lambda.gl}] Penalization parameter lambda, for the graphical lasso

\item[\code{gamma}] Gamma value for EBIC calculation of the log-likelihood
\end{ldescription}
\end{Arguments}
\inputencoding{utf8}
\HeaderA{mlePath}{Maximum Likelihood Estimator Paths}{mlePath}
%
\begin{Description}\relax
Calculates the maximum likelihood estimates of the parameters for various 'paths'
of the penalization parameter \code{lambda}.
\end{Description}
%
\begin{Usage}
\begin{verbatim}
mlePath(
  y,
  max.iter = 10000,
  max.iter.nr = 100,
  tol = 1e-06,
  tol.nr = 1e-06,
  lambda.gl = NULL,
  lambda.min.ratio = 0.1,
  n.lambda = 1,
  n.cores = NULL,
  gamma = 0.1
)
\end{verbatim}
\end{Usage}
%
\begin{Arguments}
\begin{ldescription}
\item[\code{y}] Compositional dataset

\item[\code{max.iter}] Maximum number of iterations

\item[\code{max.iter.nr}] Maximum number of Newton-Raphson iterations

\item[\code{tol}] Stopping rule

\item[\code{tol.nr}] Stopping rule for the Newton Raphson algorithm

\item[\code{lambda.gl}] Vector of penalization parameters lambda, for graphical lasso

\item[\code{lambda.min.ratio}] Minimum lambda ratio of the maximum lambda, 
used for the sequence of lambdas

\item[\code{n.lambda}] Number of lambda to evaluate differnet paths for

\item[\code{n.cores}] Number of cores to use for parallel computation

\item[\code{gamma}] Gamma value for EBIC calculation of the log-likelihood
\end{ldescription}
\end{Arguments}
%
\begin{Value}
The MLE estimates of y for each element lambda of lambda.gl, (\code{est}); 
the value of the estimates which produce the minimum EBIC, (\code{est.min}); 
the vector of lambdas used for graphical lasso, (\code{lambda.gl}); the index of 
the minimum EBIC (extended Bayesian information criterion), (\code{min.idx}); 
vector containg the EBIC for each lambda, (\code{ebic}).
\end{Value}
\inputencoding{utf8}
\HeaderA{naiveVariation}{Naive Variation}{naiveVariation}
%
\begin{Description}\relax
Estimates the variation matrix of compositional data based strictly on the counts.
\end{Description}
%
\begin{Usage}
\begin{verbatim}
naiveVariation(
  counts,
  pseudo.count = 0,
  type = c("standard", "phi", "phis", "rho", "logx"),
  use = "everything",
  set.inf.na = TRUE,
  already.log = FALSE
)
\end{verbatim}
\end{Usage}
%
\begin{Arguments}
\begin{ldescription}
\item[\code{counts}] Dataset of compositional data

\item[\code{pseudo.count}] Scaler value added to the data matrix to prevent infinite 
values caused by taking the log of the counts

\item[\code{type}] Type of variation metric to be calculated: \code{standard}, \code{phi},
\code{phis} (a symmetrical version of \code{phi}), \code{rho}, or \code{logx}

\item[\code{use}] If equal to \code{"everything"} and there are no infinite values after
taking the log the calculation will use all data. If equal to \code{"everything"} 
and there are infinite values after taking the log it is recommended to  run 
the function again, instead setting the \code{use} parameter to \code{"pairwise.complete.obs"}

\item[\code{set.inf.na}] If \code{TRUE}, sets any infinite values in \code{counts} to \code{NA}

\item[\code{already.log}] If \code{FALSE}, the counts have not been transformed by 
by the log. This transformation is of the form is \eqn{log(frac{X_{ij}}{s_{i}})}{}, where 
\eqn{s_{i} = \sum{n=1}^{j} X_{in}}{}, where \eqn{X_{ij}}{} is element \eqn{counts[i,j]}{}
\end{ldescription}
\end{Arguments}
%
\begin{Value}
The naive variation matrix, \code{v}.
\end{Value}
\inputencoding{utf8}
\HeaderA{pluginVariation}{Plugin Variation}{pluginVariation}
%
\begin{Description}\relax
Estimates the variation matrix for empirical compositional data, ?plugin.
\end{Description}
%
\begin{Usage}
\begin{verbatim}
pluginVariation(counts)
\end{verbatim}
\end{Usage}
%
\begin{Arguments}
\begin{ldescription}
\item[\code{counts}] Dataset of compositional data
\end{ldescription}
\end{Arguments}
%
\begin{Value}
\code{NULL}
\end{Value}
\inputencoding{utf8}
\HeaderA{rmse}{Root Mean Square Error}{rmse}
%
\begin{Description}\relax
Calculates the root mean square error (RMSE) between two values (i.e. scalers, 
vectors, or matrices).
\end{Description}
%
\begin{Usage}
\begin{verbatim}
rmse(x, y)
\end{verbatim}
\end{Usage}
%
\begin{Arguments}
\begin{ldescription}
\item[\code{x}] Value one for comparison (a scaler, vector, or matrix)

\item[\code{y}] Value two for comparison (a scaler, vector, or matrix). Must be the 
same form as \code{x}
\end{ldescription}
\end{Arguments}
%
\begin{Value}
A single scaler value, the RMSE.
\end{Value}
%
\begin{Examples}
\begin{ExampleCode}
x <- sample(1:20, size = 4)
y <- sample(1:20, size = 4)

rmse(x,y)

\end{ExampleCode}
\end{Examples}
\inputencoding{utf8}
\HeaderA{rmse\_by\_row}{Row Root Mean Square Error}{rmse.Rul.by.Rul.row}
%
\begin{Description}\relax
Calculates the root mean square error (RMSE) between each row of two matrices.
\end{Description}
%
\begin{Usage}
\begin{verbatim}
rmse_by_row(x, y)
\end{verbatim}
\end{Usage}
%
\begin{Arguments}
\begin{ldescription}
\item[\code{x}] First matrix for comparison

\item[\code{y}] Second matrix for comparison. Must be the same dimensions as \code{x}
\end{ldescription}
\end{Arguments}
%
\begin{Value}
A vector. Element \code{i} of the resultant vector is the RMSE which 
compares row \code{i} of matrix \code{x}, and row \code{i} of matrix \code{y}.
\end{Value}
%
\begin{Examples}
\begin{ExampleCode}
x <- matrix(sample(1:500, size = 21), nrow=3)
y <- matrix(sample(1:500, size = 21), nrow=3)

rmse(x,y)

\end{ExampleCode}
\end{Examples}
\inputencoding{utf8}
\HeaderA{varEst}{Estimated Variation Matrix}{varEst}
%
\begin{Description}\relax
Estimates the variation matrix of compositional count data.
\end{Description}
%
\begin{Usage}
\begin{verbatim}
varEst(
  counts,
  p.model = c("logitNormal", "dirichlet", "plugin"),
  type = c("standard", "phi", "rho"),
  refColumn = NULL
)
\end{verbatim}
\end{Usage}
%
\begin{Arguments}
\begin{ldescription}
\item[\code{counts}] Dataset of compositional data

\item[\code{p.model}] Probability model for the counts: (\code{logitNormal}, \code{dirichlet},
or \code{plugin})

\item[\code{type}] Type of variation metric to be calculated: \code{standard}, 
\code{phi}, or \code{rho}

\item[\code{refColumn}] The reference column to be used for the counts matrix
\end{ldescription}
\end{Arguments}
%
\begin{Value}
The estimated variation matrix for the counts. May be a \code{standard},
\code{phi}, or \code{rho} variation matrix depending on \code{type} specified.
\end{Value}
\inputencoding{utf8}
\HeaderA{wrapMLE}{Wrapper for MleLR()}{wrapMLE}
%
\begin{Description}\relax
Executes a function call to \code{mlePath()}. Helps with the \code{mclapply()} 
within \code{mlePath()}
\end{Description}
%
\begin{Usage}
\begin{verbatim}
wrapMLE(x)
\end{verbatim}
\end{Usage}
%
\begin{Arguments}
\begin{ldescription}
\item[\code{x}] Input argument for the \code{MLE Function}
\end{ldescription}
\end{Arguments}
\printindex{}
\end{document}
